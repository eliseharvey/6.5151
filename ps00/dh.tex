\input ./65150mac.tex
\def\fbox#1{%
  \vtop{\vbox{\hrule%
              \hbox{\vrule\kern3pt%
                    \vtop{\vbox{\kern3pt#1}\kern3pt}%
                    \kern3pt\vrule}}%
        \hrule}}

\def\emptybox#1#2{\fbox{\vbox to #2{\hbox to #1 {\null}}}}

\begin{document}


\psetheader{Spring 2025}{Pset 0}

{Issued: 3 February 2025} \hfill {Due: 14 February 2025}

\noindent
Reading: \\
\noindent
http://groups.csail.mit.edu/mac/users/gjs/6.5150/dont-panic/\\
{\em Software Design for Flexibility} Prologue; Appendix on Scheme\\
{\em SICP} Chapter 1; SICP Section 2.1; {\tt p0utils.scm}

\section*{General Instructions}

A problem set for this class generally asks you to do some
programming.  We usually give you a program to read and extend.  You
should turn in your extensions, in the form of a paper illustrated
with clearly annotated code and executions that demonstrate its
effectiveness.  We may also ask for short essays explaining an idea.
Your answers to these questions must be clear and concise: good logic
expressed in good English is required.  Thank you.

The purpose of this problem set is to help you get started.  Since we
will mostly use MIT/GNU Scheme you should install MIT/GNU Scheme on
your favorite computer.  Scheme is available at {\tt
http://www.gnu.org/software/mit-scheme/}.  For a bit of help look at
the ``Don't Panic'' memo, referred to by URL in the reading.  You will
also need to get experience with EDWIN, the EMACS-derived editor that
is included in MIT/GNU Scheme.  Unless you are very familiar with
EMACS we highly recommend that you go through the tutorial that comes
with the editor.  (A small amount of boring work that will save you
enormous amounts of time later.)  The exercises in this problem set
are designed to help you gain experience with writing and testing
procedures and data abstractions in Scheme.

For each problem below, include your code (with identification of the
problem number being solved), as well as comments and explanations of
your code, and demonstrate your code's functionality against a set of
test cases.  On occasion, we may provide some example test cases, but
you should {\bf always} create and include your own additional,
meaningful test cases to ensure that your code works not only on
typical inputs, but also on more difficult cases.  Get in the habit of
writing and running these test cases after {\bf every} procedure you
write---no matter how trivial the procedure may seem to you.
 
Read the entire description before you start working.  In this problem
set {\bf every sentence in boldface} describes something that you
should include in your paper---e.g., procedures to write, test cases
to run, or questions to answer.  Usually the problem sets will be less
formal than this one, but we will hold your hand here to help you get
started.

For this problem set, you need to make a directory (folder, in modern
usage) on your computer for the problem set (I would personally call
it {\tt "$\sim$/6.5150/ps00/"}) and a subdirectory called {\tt "code"}
(so you would have the directory {\tt "$\sim$/6.5150/ps00/code/"}.
Download the problem set text into {\tt "$\sim$/6.5150/ps00/"} and the
code into {\tt "$\sim$/6.5150/ps00/code/"}.  There will be a file in
the code directory named {\tt "load.scm"}.  You can load the code you
need into the Scheme system by pointing the Scheme at the code
directory, with executing {\tt (cd "$\sim$/6.5150/ps00/code/")} and
then executing {\tt (load "load")}.  The code files you load will
contain some extra material, not in the text, that support the problem
set.

For example, this problem set includes some provided code in the file
{\tt "p0utils.scm"}.

For later problem sets we will be using a more sophisiticated software
manager system for getting the code that you need and loading it into
the Scheme system.  This manager is introduced in Appendix A of the
textbook (Software Design for Flexibility).  You will not need it for
this problem set.


Note: Loading of the support files for this problem set by loading
{\tt "load.scm"} reinitializes the top-level environment of Scheme, so you
will lose any definitions you have made in that environment.  But this
will not lose buffers in your EMACS or EDWIN, so your code is not
lost.  Of course, you should write out the files you are building
regularly, probably in your directory {\tt "$\sim$/6.5150/ps00/"}.




\newpage

\section*{Introduction}

One of the longstanding problems in using encryption to encode
messages is that the recipient of the message needs to know the key in
order to decrypt the message. Clearly we somehow have to get the key
to the participants so they can use it.  We can't send the key to them
without encrypting *it*, or someone might eavesdrop and get it.  But
this puts us in an infinite loop: the person getting the key will need
to know what second key was used to encrypt the first key, etc.

How can we break this cycle? One amazingly clever idea is called the
Diffie-Hellman key agreement protocol.%
\footnote{This discovery was the greatest advance in cryptography in
  2000 years!  See the breakthrough paper:
  Whitfield Diffie and Martin E. Hellman, ``New Directions in
  Cryptography,'' in {\em IEEE Transactions on Information Theory},
  vol. {\bf IT-22}, pp 644-654, November, 1976.  Apparently, this idea
  had been thought of a bit earlier, by members of the British
  intelligence service, but they did not publish it.  Tough!}
This protocol enables two agents to share a secret, using completely open
communication: Alyssa and Ben,%
\footnote{Traditionally, in cryptographic literature, the participants
  are named Alice and Bob, but in our class they are named Alyssa and Ben.}
who have never interacted before, have
an open conversation---anyone, including Eve, the eavesdropper, may
hear the complete conversation.  At the end of the conversation Alyssa
and Ben will possess a shared secret that no one else, including Eve,
can know.  How is this possible?

The idea is very simple.  It is amazing that it had not been thought
of hundreds of years ago.  Here is the idea:
%\samepage
\begin{itemize}
\item Everyone knows two publicly advertised numbers $a$, $p$.%
\footnote{For this to work $p$ must be a prime number.  Systems based
  on some primes are harder to crack than others.  The primes that are
  usually chosen are of the form $p=2q+1$, where $q$ is itself a
  prime.  Such a $p$ is called a {\em safe prime}.  The value of the
  ``primitive root'', $a$, is usually $2$ or $5$.}
\item Alyssa chooses a secret number, $S_a$, that she remembers.
\item Ben chooses a secret number, $S_b$, that he remembers.
\item Alyssa computes $P_a = a^{S_a}\  (\bmod p)$ and announces $P_a$.
\item Ben computes $P_b = a^{S_b}\  (\bmod p)$ and announces $P_b$.
\item Alyssa computes $x = {P_b}^{S_a}\  (\bmod p) = a^{S_b S_a}\  (\bmod p)$.
\item Ben computes $y = {P_a}^{S_b}\  (\bmod p) = a^{S_a S_b}\  (\bmod p)$.
\item But ${x=y}$, so Alyssa and Ben share a secret.
\end{itemize}
The secret shared by Alyssa and Ben is the number $x=y$.
The reason why Eve cannot obtain the same secret is that for a
sufficiently large prime number $p$, and sufficiently large secret
numbers $S_a$ and $S_b$, given the public information, $a$, $p$,
$a^{S_a}\  (\bmod p)$, and $a^{S_b}\  (\bmod p)$, there is no efficient
algorithm known to get $a^{S_a S_b}\  (\bmod p)$.%
\footnote{The only known algorithm for cracking this depends on 
  the {\em discrete logarithm problem}.  As far as we know,
  if $p$, $S_a$, and $S_b$ are each only a few thousand decimal digits
  long, the time to crack this problem with classical computers,
  however fast, is longer than the lifetime of the Earth.
  But, if quantum computers can be built, cryptosystems like this one 
  will be cracked easily.  This was shown by Peter Shor in his
  landmark paper ``Polynomial-time algorithms for prime factorization
  and discrete logarithms on a quantum computer,'' which appeared in
  the {\em SIAM Journal of Computing}, {\bf 26}, pp. 1484-1509 (1997).}
The secret shared by Alyssa and Ben can then be used as a key for a
cryptographic system to encrypt and decrypt messages that cannot
be read by Eve.

In this project you will implement a Diffie-Hellman key agreement
protocol and a very simple cryptosystem based on the Diffie-Hellman
idea.  Along the way, you will have to create procedural and data
abstractions for solving some subproblems, including:
\begin{itemize}
\item modular arithmetic (addition, subtraction, and multiplication mod $n$)
\item fast exponentiation (computing $a^n$ when $n$ is very large)
\item generating large random numbers
\item testing whether a large number is prime
\item picking a large prime number at random
\item finding multiplicative inverses mod $n$
\end{itemize}

\section*{Problem 1: Modular Arithmetic}

We will need operators for {\em modular arithmetic}.
Modular arithmetic is arithmetic on a reduced set of integers, in the
range $0$ to $n-1$ for some fixed integer $n > 0$.  The integer $n$ is
called the {\it modulus}.  To indicate that we are doing modular
arithmetic, we write (mod $n$) after an expression: for example, $5+8$
(mod $12$).

Addition, subtraction, and multiplication work mostly the same way as
in integer arithmetic, except that the result must always be in the
range $[0,n-1]$.  We guarantee this by taking the remainder of the
result after dividing by $n$.  For example, $5+8=13$ in integer
arithmetic, but in mod-12 arithmetic, we take the remainder after
dividing 13 by 12, which is 1.  Here are some other examples:
\begin{eqnarray*}
5 + 8 & = & 1 \pmod{12}\\
2 + 3 & = & 5 \pmod{12}\\
6 * 5 & = & 6 \pmod{12}\\
9 - 18 & = & 3 \pmod{12}
\end{eqnarray*}
The last example may be somewhat mysterious, since $9-18=-9$ in
ordinary integer arithmetic.  To determine the correct value of
$-9$ (mod 12), we need to add or subtract a multiple of 12 that
produces a result in the desired range $[0,11]$.  More formally, we
need to find integers $a$ and $b$ such that $-9 = 12a + b$, where $0
\leq b \leq 11$.  By choosing $a = -1$, we have $-9 = -12 + b$ which
solves for $b = 3$.

Scheme has two operators for computing remainders after
division: {\tt remainder} and {\tt modulo}.  {\bf Try applying each
operator to some integers.}  For example:
%\samepage
\begin{verbatim}
(modulo 13 8)      ; -> ?
(remainder 13 8)   ; -> ?
(modulo -13 8)     ; -> ?
(remainder -13 8)  ; -> ?
(modulo -13 -8)    ; -> ?
(remainder -13 -8) ; -> ?
\end{verbatim}
{\bf What is the difference between {\tt remainder} and {\tt modulo}?
Which one is the best choice for implementing modular arithmetic as
described above?}  Include your test results and your answers
to these questions in a comment in your solution.

{\bf Write procedures for addition, subtraction, and multiplication modulo
$n$.}  Each procedure should take three parameters: the values to
combine, $a$ and $b$, and the modulus $n$.  For example, the
expression {\tt (+mod 7 5 8)} should compute $7+5$ (mod 8).  Here are 
three skeleton procedures to get you started:
%\samepage
\begin{verbatim}
(define +mod            (define -mod            (define *mod
  (lambda (a b n)         (lambda (a b n)         (lambda (a b n)
    YOUR-CODE-HERE))        YOUR-CODE-HERE))        YOUR-CODE-HERE))
\end{verbatim}

%\newpage

{\bf Test your code for at least the following cases.}

%\samepage
\begin{verbatim}
(+mod 7 5 8)     ; -> 4
(+mod 10 10 3)   ; -> 2
(-mod 5 12 2)    ; -> 1
(*mod 6 6 9)     ; -> 0
(+mod 99 99 100) ; -> ?
(*mod 50 -3 100) ; -> ?
\end{verbatim}

Notice that the procedures you wrote for modular addition, subtraction, and
multiplication are almost all the same.  Time for abstraction!  Also,
when working in modular arithmetic we often do most of our work in a
particular modulus.  Thus it is helpful to be able to define
particular operators for each modulus.  For example, we might define
addition, subtraction and multiplication modulo 12 as follows:
%\samepage
\begin{verbatim}
(define +m12 (modular 12 +))
(define -m12 (modular 12 -))
(define *m12 (modular 12 *))
\end{verbatim}
And we would be able to use these operators:
\begin{verbatim}
(-m12 (*m12 (+m12 5 8) 3) 7) ; -> 8
\end{verbatim}

{\bf Fill in the following code fragment to complete the procedure
{\tt modular} that allows us to make the binary modular operators.
{\tt modular} should take a modulus and an ordinary arithmetic
procedure and produce a modular arithmetic procedure.  For example,
{\tt ((modular 8 +) 7 5)} should compute $(7+5)\pmod{8} = 4$.}
%\samepage
\begin{verbatim}
(define modular
  (lambda (modulus op)
    (lambda (a1 a2)
      YOUR-CODE-HERE)))
\end{verbatim}
%\newpage

{\bf Test your code for at least the following cases.}
%\samepage
\begin{verbatim}
((modular 17 +) 13 11)   ; -> 7
((modular 17 -) 13 11)   ; -> 2
((modular 17 *) 13 11)   ; -> 7
\end{verbatim}

Quite often we will use an alternative syntax for defining procedures.
The following definitions are equivalent:
%\samepage
\begin{verbatim}
(define modular
  (lambda (modulus op)
    (lambda (a1 a2)
      BODY OF THE PROCEDURE)))

(define (modular modulus op)
  (lambda (a1 a2)
    BODY OF THE PROCEDURE))

(define (modular modulus op)
  (define (the-operator a1 a2)
    BODY OF THE PROCEDURE)
  the-operator)

(define ((modular modulus op) a1 a2)
  BODY OF THE PROCEDURE)
\end{verbatim}
The choice of definition syntax is a matter of style.  You may want to
emphasize some particular aspect for the reader to notice.  After all,
much of the value of computer language is communicating to human
readers.


\subsubsection*{Note about preparing your submission}

You should run the tutorial for the editor so you understand how to
work in the MIT/GNU Scheme environment.  As you are working on your
project, you may want to plan ahead for the document that you will
submit as your work.  We assume that you will start a file for this
problem set (ps00) with some name like {\tt ps00.scm} by typing the
command {\tt C-x~C-f~ps00.scm} to EDWIN.  This will make an editing
buffer that will be in Scheme mode, helping you with balancing
parentheses and adjusting indentation.  When you save the buffer (with
{\tt C-x~C-s}) the buffer will be written out as the file with name
{\tt ps00.scm} in your directory.  You will also have a buffer named
*scheme*, which is running a read-eval-print loop.  (You can show both
buffers simultaneously as two windows and move between them easily.)
You will compose your answers to the problems in the {\tt ps00.scm}
directory.  You can comment out results of evaluation, explanations,
and discussion with comment brackets {\tt \#|} and {\tt |\#}.
When you save the file, you get something you can read back and work
on again, or print and submit for us to read.  If you comment out only
the results and the commentary text then you can execute the contents
of the file directly.  The Scheme interpreter will ignore the
commented material.

We will help you to get started.


\section*{Problem 2: Raising a Number to a Power}

{\it This problem is easy if you've read Chapter 1 in SICP.}
 
Recall that the basic operation in Diffie-Hellman key agreement is
raising a number to a power (modulo $n$).

Here's a simple procedure that computes $a^b$ (mod $n$) by multiplying
$a$ by itself $b$ times.  Note that it uses modular arithmetic
operations, namely {\tt *mod}, rather than {\tt *}:

%\samepage
\begin{verbatim}
(define (slow-exptmod n)
  (let ((*mod (modular n *)))
    (define (em a b)
      (if (= b 0)
          1
          (*mod a (em a (- b 1)))))
    em))
\end{verbatim}

Answer these questions in comments in your file: {\bf What is the
  order of growth in time of {\tt slow-exptmod}?  What is its order of
  growth in space?  Does {\tt slow-exptmod} use an iterative algorithm
  or a recursive algorithm?}  Measure time and space the same way we
did in lecture: time by counting the number of primitive operations
that the computation uses, and space by counting the maximum number of
pending operations.

As its name suggests, {\tt slow-exptmod} isn't going to be fast enough
for our purposes.  We can make a faster procedure using the trick of
{\it repeated squaring.}%
\footnote{This technique is discussed in section 1.2.4 of SICP.}
Compare these two ways of computing $3^8$.  The left column shows how
{\tt slow-exptmod} would do it, and the right column uses repeated
squaring:

\begin{verbatim}
3^0 = 1            3^0 = 1
3^1 = 3^0*3 = 3    3^1 = 1 * 3 = 3
3^2 = 3^1*3 = 9    3^2 = (3^1)*(3^1) = 9
3^3 = 3^2*3 = 27   3^4 = (3^2)*(3^2) = 81
3^4 = 3^3*3 = 81   3^8 = (3^4)*(3^4) = 6561
3^5 = 3^4*3 = 243
3^6 = 3^5*3 = 729
3^7 = 3^6*3 = 2187
3^8 = 3^7*3 = 6561
\end{verbatim}

\newpage

{\bf Fill in the details of the procedure {\tt exptmod}
     that computes $a^b $ (mod $n$) using repeated squaring.} 
     You should use your modular arithmetic operations in your
     solution. Do not use {\tt expt} or {\tt slow-exptmod} in
     your solution.

%\samepage
\begin{verbatim}
(define (exptmod p)
  (let ((mod* (modular p *)))
    (define (square x)
      (mod* x x))
    (define (em base exponent)
      YOUR-CODE-HERE)
    em))
\end{verbatim}

{\bf Test your code for at least the following cases:}
%\samepage
\begin{verbatim}
((exptmod 10) 2 0)   ; -> 1
((exptmod 10) 2 3)   ; -> 8
((exptmod 10) 3 4)   ; -> 1
((exptmod 100) 2 15) ; -> 68
((exptmod 100) -5 3) ; -> 75
\end{verbatim}

Answer these questions in comments in your file: {\bf What is the
  order of growth in time of your implementation of {\tt exptmod}?
  What is its order of growth in space?  Does your {\tt exptmod}
  use an iterative algorithm or a recursive algorithm?}


\section*{Problem 3: Large Random Numbers}

We will need a source of random numbers for the modulus and for the
individual secrets.  Scheme has a builtin procedure {\tt random} that
takes a single integer $n>0$ and returns a random integer in the range
$[0,n-1]$.  For example, here are the results of a few calls to {\tt
  random}:

%\samepage
\begin{verbatim}
(random 10) ; -> 1
(random 10) ; -> 6
(random 10) ; -> 6
(random 10) ; -> 0
(random 10) ; -> 7
\end{verbatim}

Unfortunately, the implementation of {\tt random} in some versions of
Scheme is not sufficient for our purposes, because its parameter $n$
can be no larger than $2^{31}-1$, which is only a couple billion.
(The random number generator in MIT/GNU Scheme does not have this
limitation.)  But we're going to want random numbers at least as large
as $2^{128}$, if not larger.  

Just for practice, let's proceed as if we did not have a fully
competent Scheme.  So our goal for this problem is to make a procedure
{\tt big-random} that behaves like {\tt random}, taking a parameter
$n>0$ and returning a random number in $[0,n-1]$, but that doesn't
have any limit on the size of $n$.

{\bf Start by writing a procedure {\tt random-k-digit-number} that
  takes an integer $k>0$ and returns a random $k$-digit number.}  You
should choose each digit using the builtin procedure {\tt random},
then construct the $k$-digit number by putting those digits together.
For example, if you generate two digits $a$ and $b$, then you can form
a two-digit number by computing $10a + b$.

{\bf Test your procedure on at least the following test cases:}
%\samepage
\begin{verbatim}
(random-k-digit-number 1)  ; -> ?        (1 digit)
(random-k-digit-number 3)  ; -> ???      (1-3 digits)
(random-k-digit-number 3)  ; -> ???      (is it different?)
(random-k-digit-number 50) ; -> ???...   (1-50 digits)
\end{verbatim}

Note that {\tt random-k-digit-number} may return a number shorter than
$k$ digits, since the leading digits of the number may turn out to be
$0$.  The result will be a random number in the range $[0,10^k-1]$.

With {\tt random-k-digit-number}, we can now generate arbitrarily
large random numbers.  But we want {\tt big-random} to take a maximum
$n$, not a digit count.  So we need a way to use {\tt
  random-k-digit-number} to generate random numbers in the range
$[0,n-1]$.  We'll do this by first generating a random number with the
same number of digits as $n$, then ensuring that this number is less
than $n$.

{\bf Write a procedure {\tt count-digits} that takes an integer $n>0$
  and returns the number of digits in its decimal representation.}
  One way to do this is to count digits by repeatedly dividing by 10.
  {\bf Test your procedure on at least the following test cases:}

%\samepage
\begin{verbatim}
(count-digits 3)         ; -> 1
(count-digits 2007)      ; -> 4
(count-digits 123456789) ; -> 9
\end{verbatim}

We're almost ready to write {\tt big-random}, but there's one more
problem.  Suppose somebody calls {\tt (big-random 500)}, expecting to
get back a number between 0 and 499.  We use {\tt count-digits} to
determine that 500 has 3 digits, and then use {\tt
  random-k-digit-number} to generate a random 3-digit number.  If that
number is less than 500, then great, we can return it as the result of
{\tt big-random}.  But what if the number is greater than or equal to
500?  Then we just pick another random 3-digit number.  We repeat this
process until we get a number that's in the range we want.

This is a simple example of a {\it probabilistic} algorithm---an
algorithm that depends on random chance.  A probabilistic algorithm
isn't {\it guaranteed} to succeed, but its probability of success can
be made as high as we need it to be.  In this case, it's possible for
the algorithm to have a really bad string of luck, and repeatedly pick
3-digit numbers higher than 500.  But the chance of this happening,
say, 1000 times in a row is the same as the chance of flipping heads
on a coin 1000 times in a row, which is less than the probability that
cosmic rays will cause your computer to make an error in running your
Scheme code.  So, in practice, as long as we keep picking random
numbers (and assuming {\tt random-k-digit-number} really is random),
this probabilistic algorithm is just as likely to succeed as a
deterministic algorithm.

\newpage

{\bf Use this approach to write a procedure {\tt big-random} that
  takes an integer $n>0$ and returns a random number from 0 to $n-1$.}
Your procedure should handle arbitrarily large $n$.

Since your procedure needs to generate a random number, test it for a
property, and then return it if it satisfies that property.

{\bf Test {\tt big-random} on at least these test cases:}
%\samepage
\begin{verbatim}
(big-random 100)          ; -> ??  (1-2 digit number)
(big-random 100)          ; -> ??  (is it different?)
(big-random 1)            ; -> 0
(big-random 1)            ; -> 0     (should be always 0)
(big-random (expt 10 40)) ; -> ????...  (roughly 40-digit number)
\end{verbatim}

\newpage

\section*{Problem 4: Prime Numbers}

{\it This problem is easy if you've read Chapter 1 in SICP.}
 
In the Diffie-Hellman key agreement protocol we will need a very
large prime number for the modulus p.  Let's go!

We'll start by developing a test for whether a number is prime.  By
definition, a prime number is not divisible by any integer other than
itself and 1.  This leads directly to a simple way to test whether $n$
is prime, by testing every number less than $n$ to see if it's a
factor of $n$:

%\samepage
\begin{verbatim}
(define (slow-prime? n)
  (define (test-factors n k)
    (cond ((>= k n) #t)
          ((= (remainder n k) 0) #f)
          (else (test-factors n (+ k 1)))))
  (if (< n 2)
      #f
      (test-factors n 2)))
\end{verbatim}

Answer these questions in comments in your file: {\bf What is the
  order of growth in time of {\tt slow-prime?}  What is its order of
  growth in space?  Does {\tt slow-prime?} use an iterative algorithm
  or a recursive algorithm?}

Unfortunately {\tt slow-prime?} is too slow.  Ben Bitdiddle
proposes two optimizations:
\begin{itemize}
\item ``We only have to check factors less than or equal to $\sqrt{n}$.'' 
   {\bf How would this affect the order of growth in time?}  
   Note that we're not asking you to write Scheme code
   implementing Ben's suggestion; just think about it and answer this
   question as a comment in your file.
\item ``We only have to check odd factors (and 2, as a special case).''  
   {\bf How would this affect the order of growth in time?}
\end{itemize}

Ben's improvements won't be enough for us to test very large numbers
for primality; we need a completely different algorithm.  For faster
prime number testing, we turn to a beautiful result about modular
arithmetic.  Fermat's Little Theorem states that if $p$ is prime, then
$a^p = a$ (mod $p$) for all $a$.  In other words, if $p$ is prime,
then we can take any integer $a$, raise it to the power $p$, take the
remainder after dividing by $p$, and we'll get $a$ back again (modulo
$p$).  
{\bf Test Fermat's Little Theorem using your {\tt exptmod}
     procedure and a few suitable choices of $a$ and $p$.}
  Include your tests in your answer file.

\newpage

The converse of the theorem doesn't hold, unfortunately; if $p$ is
composite (not prime), then it isn't always true that $a^p \neq a$
(mod $p$).  But it's true often enough that we can use this theorem as
the basis for a {\it probabilistic} algorithm that tests whether a
number $p$ is prime:

\begin{enumerate}
\item Pick a random integer $a$ in the range $[0,p-1]$, using
  your {\tt big-random} procedure.
\item Test whether $a^p = a$ (mod $p$).
\item If {\it not}, then $p$ is definitely not prime, by Fermat's Little Theorem.
\item If so, then $p$ may or may not be prime.  
      Repeat the test with a new random integer $a$.
\end{enumerate}

If you pick enough random numbers $a$, and all of them pass the test
of Fermat's Little Theorem, then you have strong confidence that $p$
is prime.%
\footnote{But not certainty.  Some composite numbers $p$,
  called Carmichael numbers, pass the test for almost all $a$.
  Fortunately Carmichael numbers are rare. 
  See {\tt http://mathworld.wolfram.com/CarmichaelNumber.html}
  for more information.}

{\bf Write a procedure {\tt prime?} that uses this technique
  to test whether its parameter $p$ is prime.}
  Your procedure should test at least 20 random values of $a$
  before assuming that $p$ is prime.  In fact, it's good practice 
  to define a name for this constant, {\tt prime-test-iterations}, 
  since it may need to be adjusted later.

%\samepage
\begin{verbatim}
(define prime-test-iterations 20)

(define prime?
  (lambda (p)
    YOUR-CODE-HERE))
\end{verbatim}

{\bf Test {\tt prime?} on at least the following test cases:}

%\samepage
\begin{verbatim}
(prime? 2) ; -> #t
(prime? 4) ; -> #f
(prime? 1) ; -> #f
(prime? 0) ; -> #f
(prime? 200) ; -> ?
(prime? 199) ; -> ?
\end{verbatim}

Answer these questions in comments in your file: {\bf What is the
  order of growth in time of your implementation of {\tt prime?}  What
  is its order of growth in space?}  Be sure to take the calls to {\tt
  exptmod} into account when answering these questions.  {\bf Does
  {\tt prime?} use an iterative algorithm or a recursive algorithm?}


\section*{Problem 5: Random Primes}


Fortunately, prime numbers are fairly
common,%
\footnote{Another famous result, the Prime Number Theorem, holds
  that the number of primes less than $n$ is roughly $n/\ln n$.  For
  example, if we pick a random 40-digit number, then the probability
  that it's prime is roughly $1/\ln 10^{40}$, or $1/92$.  See {\tt
    http://mathworld.wolfram.com/PrimeNumberTheorem.html}.}
so we can find them by a probabilistic {\it generate and test}
strategy. We'll guess a number at random, and then test whether it's 
prime.  If not, we'll pick another random number, and repeat.

{\bf Write a procedure {\tt random-k-digit-prime} that returns a random prime
  number with about $k$ digits.}  

%\samepage
\begin{verbatim}
(define random-k-digit-prime
  (lambda (k)
    YOUR-CODE-HERE))
\end{verbatim}

Note that your procedure is {\it probabilistic} -- i.e., most of the
time it successfully returns a prime number, but sometimes it may
fail.  {\bf In what ways can your random-prime procedure fail?}
Answer this question in a comment. Some kinds of failure are better
than others.

{\bf Test {\tt random-k-digit-prime} on at least the test cases below.}

%\samepage
\begin{verbatim}
(random-k-digit-prime 1)
(random-k-digit-prime 2)    
(random-k-digit-prime 10)
(count-digits (random-k-digit-prime 100))   ; Not always 100.
(count-digits (random-k-digit-prime 100))
\end{verbatim}

\section*{Problem 6: Multiplicative Inverses}

We will also need a way to find {\it multiplicative inverses} in
modular arithmetic.  Given an integer $e$ and a modulus $n$, we want
to find $d$ such that $ed = 1 $ (mod $n$).  In rational or real
arithmetic, $d$ would be $1/e$, but we want an integer.  The
multiplicative inverse of $e$ exists if and only if $e$ and $n$ have
no common factors; in other words, only if the greatest common divisor
(GCD) of $e$ and $n$ is 1.  (The GCD algorithm is described in section
1.2.5 of the text, but you can use the Scheme builtin procedure {\tt
gcd} for this project.)

Here's how we find the multiplicative inverse $d$.  We want $ed=1$
(mod $n$), which means that $ed + nk = 1$ for some integer $k$.  So
we'll write a procedure that solves the general equation $ax + by =
1$, where $a$ and $b$ are given, $x$ and $y$ are variables, and all of
these values are integers.  We'll use this procedure to solve $ed + nk
= 1$ for $d$ and $k$.  Then we can throw away $k$ and simply return
$d$.

So we've reduced the problem to solving $ax+by=1$ for $x$ and $y$,
given $a$ and $b$.  We assume that all of these terms are integers,
and that $a$ and $b$ are greater than 0.  Let $q$ be the quotient of
dividing $a$ by $b$, and let $r$ be the remainder.  (Scheme has
builtin procedures {\tt quotient} and {\tt remainder} for this
purpose.)  Then $a = qb + r$.  Now consider the special case when
$r=1$: then $a=qb+1$, which means $a \cdot 1 + b(-q)=1$, so we have
our solution; $x=1$ and $y=-q$.  Otherwise, if $r\neq 1$, recursively solve the equation
$bx'+ry'=1$, and use the solution $(x',y')$ to find the solution to
the original equation $ax+by=1$:
$$1 = bx' + ry' = bx' + (a-qb)y' = ay' + b(x'-qy')$$

{\bf Write a procedure {\tt ax+by=1} that solves for $x$ and $y$ using
  the approach outlined above.}  Your procedure should return $(x,y)$
as a list.

%\samepage
\begin{verbatim}
(define ax+by=1
  (lambda (a b)
    YOUR-CODE-HERE))
\end{verbatim}

{\bf Test {\tt ax+by=1} on at least the test cases below.}  Note that
it will only succeed if $\gcd(a,b)=1$, so don't expect it to work
otherwise.

%\samepage
\begin{verbatim}
(ax+by=1 17 13) ; -> (-3 4)    17*-3 + 13*4 = 1
(ax+by=1 7 3)   ; -> (1 -2)     7*1  + 3*-2 = 1
(ax+by=1 10 27) ; -> (-8 3)    10*-8 + 3*27 = 1
\end{verbatim}

{\bf Now write a procedure {\tt inversemod} that finds the
  multiplicative inverse of $e$ modulo $n$, using {\tt ax+by=1}.}
Note that before trying to invert $e$, your procedure should ensure
that $\gcd(e,n)=1$.  You can use the builtin Scheme procedure {\tt
  gcd} to test this, and the Scheme builtin procedure {\tt error} to
signal an error if the test fails.

%\samepage
\begin{verbatim}
(define (inversemod n)
  (lambda (e)
    YOUR-CODE-HERE))
\end{verbatim}

{\bf Test {\tt inversemod} on at least the test cases below.}

%\samepage
\begin{verbatim}
((inversemod 11) 5) ; -> 9            5*9 = 45 = 1 (mod 11) 
((inversemod 11) 9) ; -> 5
((inversemod 11) 7) ; -> 8            7*8 = 56 = 1 (mod 11)
((inversemod 12) 5) ; -> 5            5*5 = 25 = 1 (mod 12)
((inversemod 12) 8) ; -> error        gcd(8,12)=4, so no inverse exists
((inversemod 101) (random-k-digit-prime 2))
         -> ? (test your answer with *mod)
\end{verbatim}


\section*{Problem 7: The ElGamal Public-Key Cryptosystem}

The ElGamal public-key cryptosystem%
\footnote{See: Taher ElGamal, 
  ``A Public-Key Cryptosystem and a Signature Scheme Based on 
    Discrete Logarithms,'' 
    in {\em IEEE Transactions on Information Theory},
    vol. {\bf IT-31}, no. 4, pp. 469--472, 1985.}
is based on the Diffie-Hellman key-agreement protocol.  It is used in PGP%
\footnote{PGP was invented by Phil Zimmerman.  It is best described in
  Simson Garfinkel, 
  {\bf PGP: Pretty Good Privacy}, O'Reilly \& Associates.}
and other common cryptography applications.

A public-key cryptosystem does not use the same key for both
encrypting and decrypting messages. There are two keys: a {\em public}
key used for encrypting, and a secret {\em private} key for
decrypting.  The public key can be known by anybody; in fact, it's
often put in a public directory, so that Alyssa can send secret
messages to Ben simply by looking up his public key in the directory.
His private key $d$ must be known only to Ben, so that nobody else can
decrypt messages sent to him.  Your MIT browser certificate contains a
public-key/private-key pair.

Here is how the ElGamal system works: Each receiver chooses a secret
private key, $S$.  The receiver publishes a public key, which gives
$a$, $p$, and $P=a^S\  (\bmod p)$.  The receiver also supplies a procedure
that a sender can call with the ciphertext.  Given a message, $m$,
the sender chooses his own secret, $T$, and sends a ciphertext that
has two components, $(x, y)$, where $x=a^T\  (\bmod p)$ and $y=m P^T
\  (\bmod p)$.  The Diffie-Hellman shared secret is $x^S\  (\bmod p)=a^{S
T}\  (\bmod p)=P^T\  (\bmod p)$.  So the receiver decrypts the message by
computing $m=y (x^S)^{-1}\  (\bmod p)$.

In the file {\tt p0utils.scm} you will find code for making a receiver:
\begin{verbatim}
(define (eg-receiver dh-system)
  (let ((k (dh-system-size dh-system))
        (p (dh-system-prime dh-system)))
    (let ((my-secret (random-k-digit-number k))
          (mod-expt (exptmod p))
          (mod-* (modular p *))
          (mod-inv (inversemod p)))
      (let ((advertised-number
             (mod-expt (dh-system-primitive-root dh-system) my-secret)))
        (let ((public-key
               (eg-make-public-key dh-system advertised-number))
              (decryption-procedure
               (lambda (ciphertext)
                 (let ((x (eg-ciphertext-x ciphertext))
                       (y (eg-ciphertext-y ciphertext)))
                   (let ((m (mod-* y (mod-inv (mod-expt x my-secret)))))
                     (integer->string m))))))
          (eg-make-receiver public-key decryption-procedure))))))
\end{verbatim}

This program is intended to work with a procedure that you will write
that sends messages to the receiver:
\begin{verbatim}
(define (eg-send-message message receiver)
  YOUR CODE HERE)
\end{verbatim}
An example of the way this pair of procedures can be used is:
\begin{verbatim}
(define dh-system (public-dh-system 100))

(define Alyssa (eg-receiver dh-system))

(eg-send-message "Hi there." Alyssa)
;Value: "Hi there."
\end{verbatim}

In the file {\tt p0utils.scm} you will also find definitions of all of the
data abstractions that you will need for this exercise, including the
code that translates strings to integers and integers to strings.
For example:

%\samepage
\begin{verbatim}
(string->integer "hello") ; -> 1578072040808
(integer->string 1578072040808) ; -> "hello"

(string->integer "") ; -> 1
(integer->string 1) ; -> ""
\end{verbatim}

In order to use the provided code in your own file without having to
copy and paste it, put the provided file {\tt p0utils.scm} in the same
folder as your own file and add the expression {\tt (load "p0utils.scm")}
at the top of your file.  The {\tt load} procedure evaluates all the
definitions and expressions in {\tt p0utils.scm}, so that you can use {\tt
string->integer} and {\tt integer->string} in your own code.

We represent the message we want to send as an integer.%
\footnote{There are several ways
  to do this.  One way is to encrypt one byte of the message at a 
  time, using its numeric value as the integer
  representation.  Another way is to treat the bytes of the
  message as digits in a base-256 number, so that the entire 
  message becomes the integer.
  We'll use the latter approach in this project.  Practical
  cryptosystems do
  something in between, converting chunks of the message as integers and
  encrypting one chunk at a time.}  

{\bf Write a procedure {\tt eg-send-message} that takes a message
string and a receiver and calls the receiver's decryption procedure,
encrypted with the receiver's public key.}  The receiver we provided
will decrypt the original message and return the string, if your
sender is correctly written.

{\bf Demonstrate your {\tt eg-send-message} with a few short strings
sent from Ben to Alyssa.}

{\bf What is the longest string you can send that will be correctly
decrypted with a 100 digit system?  You will find that it is not too
long!}

\section*{Problem 8: ``Man In The Middle'' Attack}

One way we can spy on a cryptosystem is by a "man in the middle"
attack.  For example, Eve can shadow Alyssa and observe all of the
messages sent to her using the following code.
\begin{verbatim}
(define (Eve receiver)
  (let ((receiver-public-key
         (eg-receiver-public-key receiver))
        (receiver-decryption-procedure
         (eg-receiver-decryption-procedure receiver)))
    (let ((my-spying-procedure
           (lambda (ciphertext)
             (write ciphertext)
             (newline)
             (receiver-decryption-procedure ciphertext))))
      (eg-make-receiver receiver-public-key
                        my-spying-procedure))))
\end{verbatim}
We provide you with this Eve code.  You can try it as follows.  After
defining Alyssa, execute:
\begin{verbatim}
(define Alyssa (Eve Alyssa))
\end{verbatim}
Thus Eve will catch and print every message that Ben tries to send to Alyssa.
However, Eve will not be able to interpret the message, because it is
encrypted.  However, Eve can still make trouble for the Ben-Alyssa
relationship using this trick.  How?

{\bf Modify the {\tt Eve} program to make it possible for Eve to make
trouble in the relationship.  Explain and demonstrate your nasty
trick.}

\end{document}
